\section{introduction}
\label{sec:introduction}
Opportunistic Networks (OppNets) have emerged to provide the infrastructure-less scenarios, 
such as underwater, UAV, rural areas, battle fields,
%ref
with the low-cost message dissemination service,
and the supplement of the cellular network, including 5G.
Due to the intermittent connectivity,
OppNets rely on the {\it store-carry-forward} paradigm,
where the forwarded messages will be stored in the intermediate node 
until delivered to the destination or better relay nodes upon the contacts.

In order to decide whether to forward or store the message when the contact occurs,
a lot of opportunistic routing algorithms have been proposed.
%ref
The static routing algorithms, including Epidemic, DirectRouting and S\&W, 
treat all the intermediate node equally
and do not make use of the relation among nodes. 
Serval routing algorithms based on the social attributes
exploited the similarity evaluation to selectively forward messages to the nodes,
which is similar with the destination nodes.
The contact history is the important metric to measure the probability 
that the relay node can contact the destination node.
Moreover, the location and the link state 
is also the auxiliary information of the routing decision.

In many existing works on opportunistic routing,
the contacts between nodes are assumed as conforming to the Poisson process assumption.

which can be improved with the accurate probability 


In this paper, we focus on the opportunistic routing in the pattern scenario, i.e., the bike sharing system.

The main contributions are as follows.
\begin{itemize}
\item {we designed the two-tier probability model
to profile the probability density of the future contact events
in the message lifetime.}
\item {we employed the inter-day probability based on the poisson process
the intra-day probability pattern based on gaussian mixture model
to measure the probability of sequential events.
Multiple potential routes of message forwarding
are utilized to support the routing decision.}
\item {we exploit the local differential privacy
to obscure the pattern of contact events.}
\item {we evaluate the proposed scheme with the benchmarks,
such as S\&W, Epidemic, Prophet,
in terms of the routing performance under various settings
with the bike sharing trace.}
\end{itemize}

The rest of this paper is as follows.
Section~\ref{sec:relatedwork} reviews the literatures about OppNets.
Section 
