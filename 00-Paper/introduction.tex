\section{introduction}
\label{sec:introduction}
Opportunistic Network (OppNet) has been
one of the most popular solutions to {\it the data dissemination in the infrastructure-less scenarios},
including Unmanned Aerial Vehicles~\cite{uav20iecommg, fanet19iecommg},
Underwater Acoustic Sensor~\cite{underwater21tvt, Acou18ieNetw, thrO20cn}, isolated/disaster areas~\cite{disaster20tsmcs, rural21wcmc},
and {\it the supplement of the existing cellular networks},
e.g., the data offloading~\cite{freshoffload20twc, offload21sensors},
in the recent years.
Due to the lack of permanent end-to-end data delivery routes,
OppNets always rely on the {\it store-carry-forward} paradigm,
where the forwarded messages will be stored in the intermediate nodes
until be delivered to the destination or the better relay nodes upon next intermittent contacts.

To improve the efficiency of the message dissemination,
the auxiliary information,
such as the spatial-temporal features~\cite{losero20tmc}, the contact history/pattern~\cite{dataInten19infocom, sporadic20wcnc},
the social attributes~\cite{loweng21ppna} and the channel states~\cite{csi20jsac},
is utilized in the design of the opportunistic routing,
which decides whether or not to forward/replicate the message to the contacted node.
%Meanwhile, the theory analysis of OppNets performance
%is usually based on the assumption,
%where the node contacts conform to the poisson processes.
%Recently, machine learning is adopted in the opportunistic routing schemes design,
%including SVM, GMM, decision tree, neural network
%and Q-learning.
%However, non-Poisson (GMM) two-prob model
However, computing the routing metric requires the auxiliary information exchange among nodes,
which may leads to the privacy disclosure of the node (even the user).
In order to preserve the privacy of the participating node in OppNet,
some research exploited the homomorphic encryption~\cite{pp16tvt,ppoct18tvt}
and $k$-anonymity~\cite{face17ton,pploc18iotj}.
The encryption techniques still can not avoid the risk
that the malicious user infer the historical contact of a node,
which represents the mobility of the corresponding person.
Differential Privacy (DP), as the {\it de facto} standard for data privacy,
has been adopted in many application fields,
for example, the statistics of client software in Google Inc.~\cite{rappor14ccs}
%the data collection in  
and the data collections of popular emojis in Apple Inc.~\cite{2017Emoji}.
%%Motivation-2
%Another burning issues plaguing OppNets today is the privacy concern.
%The routing auxiliary information exchanged among nodes
%may expose the features of nodes
%and even determine the identity of users in the restricted scenarios.
%However, few existing privacy protections in OppNets
%take the routing auxiliary information into account.
%Since the Differential Privacy (DP) has been accepted
%as the {\it de facto} standard for data privacy,
%we adopt the differential privacy techniques
%in the routing auxiliary information exchange
%to provide the privacy protection.

In this paper, we focus on the data dissemination 
among bike stations of the bike-sharing system,
where the message can be forwarded when the bicycle trip occurs.  
%in the contact circumstance of non-poisson process, i.e., .
Considering that the temporal pattern of the bicycle trip occurring 
is different from the classical poisson process,
we proposed the two-tier probability model
to predict the future contact based on the history records.
%to capture the multi-hop forwarding opportunities
Moreover, differential privacy is exploited in the auxiliary information exchange
to achieve less privacy exposure.
The main contributions are as follows.
\begin{itemize}
\item {we designed the two-tier probability model,
i.e., the inter-day probability and the intra-day probability,
to profile the probability density of the future contact occurring.}
%we employed the inter-day probability based on the poisson process
%the intra-day probability pattern based on gaussian mixture model
%to measure the probability of sequential events.}
\item {we exploited local differential privacy and the laplace mechanism 
to obscure the published pattern of node contacts and preserve the privacy of the node.}
\item {we converted the delivery metric based on probability density integral into the additive one,
and proposed the routing scheme with differential privacy.}
\item {we evaluate the proposed scheme with the benchmarks,
such as S\&W, Epidemic, Prophet,
in terms of the routing performance under various settings
with the real-world bike sharing trace.}
\end{itemize}

The rest of this paper is as follows.
Section~\ref{sec:relatedwork} reviews the literatures about OppNets.
We model the contact behavior and introduce the two-tier probability model
in Section~\ref{sec:sysmodel}.
The routing scheme with differential privacy protection
is described in Section~\ref{sec:method}.
The corresponding performance is evaluated in Section~\ref{sec:pe}.
We conclude the paper in Section~\ref{sec:conclusion}.

