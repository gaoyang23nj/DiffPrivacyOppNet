\section{Two-Tier Probability Model}
\label{sec:sysmodel}
We assume a set of nodes $\mathcal{N}$ in the network.
In this paper, we mainly consider the message forwarding occurs by the bike trip among the bike stations.
Thus the bike trip from node (station) $i$ to node (station) $j$ 
can provide the message forwarding opportunity from $i$ to $j$,
but can not serve the message forwarding from $j$ to $i$.
The bike trip can also be treated as the directional contact,
which is denoted by $i \rightarrow j$.

\subsection{Inter-day probability}
we assume that the contact events among the nodes are daily periodic
and follow the Poisson process.
Fig. 1** shows that the number of contacts in the range of year.
We can find the contacts are regular among the successive days
and vary slowly in the range of year.
Let $\bar{C}_{i j}$ denote the expected number of contacts from $i$ to $j$ in one day.
The probability that $i \rightarrow j$ occurs in a future day (after $k$th day) can be calculated as
\begin{small}
\begin{equation}
\begin{aligned}
p^{k}_{i j} = 1 - Prob\{\bar{C}_{i j}=0\} = 1 - exp(-C^{k}_{i j}),
\end{aligned}
\end{equation}
\end{small}
where the number of contact records in each day, e.g., $C^{k}_{i j}$ in $k$th day,
will be collected in the end of each day.

Since the inter-day pattern may vary caused by 
temperature, holiday/workday, weather and air pollution, in the range of year,
the inter-day probability of the contact occurrence, e.g., $P^{k+1}_{i j}$ in $k+1$th day,
should be updated according to the contact records in each day.
In this paper, we use the EWEA method (Exponentially Weighted Moving Average) to
model the probability that at least a contact $i \rightarrow j$ can occurs in $k+1$th day,
which is
\begin{small}
\begin{equation}
P^{k+1}_{i j} =
\left\{
\begin{aligned}
& p^{k}_{i j}, &\text{if  } k = 0 \\
& \alpha P^{k}_{i j} + (1-\alpha) p^{k}_{i j}, &\text{if  } k \le 1
\end{aligned}
\right.
\end{equation}
\end{small}
Here the weight coefficient $\alpha \in (0, 1)$.

Additionally, the inter-day probability patterns are different between the holidays and the workdays.
If $k$th day is the holiday, including weekends, May 1st (Labour Day), Mid-autumn Festival,  
$C^{k}_{i j}$ should be utilized to calculate $Pw^{k+1}_{i j}$;
otherwise $Ph^{k+1}_{i j}$.
With this method, we can calculate the probability that the contact $i \rightarrow j$ can occur 
at the beginning of each day with the contact history dataset.

\subsection{Intra-day probability}
As Fig.2** shows,
we find that the contact probability patterns in the range of day
are not same for any two node pair.
The contact history from the residence to the metro station 
show the apparent peak phenomenon,
which means that the most of contacts occur in the morning rush hour, i.e. $7$ a.m. to $9$ a.m..
But the contacts from the metro station to the residence
mainly focus in the afternoon peak hours.
Some contacts in the rural areas will not show the apparent difference between the peak and the valley.
In order to model the intra-day probability pattern,
we fit the gaussian mixture model for each pair $(i, j)$.

After the trip collection in each day, the trip time will be collected.
Similar with the inter-day probability,
