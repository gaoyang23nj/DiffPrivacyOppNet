\section{related work}
\label{sec:relatedwork}
%%% �����������ֵĴ�С 1.incentive& restrict 2.detect 3.ml
In this section, we first review the various selfish behaviors in OppNets.
Then two existing categories of blackhole mitigating methods
and the applications of machine learning in OppNets
are discussed.

\subsection{Opportunistic Routing Protocols}
To design a geo-casting routing protocol,
Costantino {\it et al.}~\cite{losero20tmc} forwarded the messages to the relay nodes,
whose most frequently visited location intersected the destination zone.
A bitmap-based routing scheme~\cite{bitmap21cn} exploited a weighted tree,
which indicates whether the destination is reachable from the specific relay node.
Since the transmitted data may be very large for one contact in OppNets,
\cite{dataInten19infocom} integrated multiple attributes,
including the contact frequency, the contact duration and buffer constraint,
and designed the path weight as the bottleneck link capacity.
The routing metric based on the real-time expected transmission count and the backpressure
is proposed in~\cite{cntawloran20icdcs} to alleviate the inference from obstacles or deep fading.
Considering that machine learning has been widespread in multiple domains,
\cite{mlrouting19Secon} designed the Cascaded Machine Learning based routing protocol (CAML),
where the two-stage classifier consisting of a Logistic Regression classifier and a Neural Network classifier.
The node intimacy and the support vector machine are utilized in~\cite{rIntimacy21wcmc}
to quantify the encounter records of opportunistic nodes.
Considering that the excessive dependence on key nodes will shorten the life cycle of the network
\cite{loweng21ppna} utilized the energy consumption model and the cache replacement strategy
to realize the low energy consumption opportunistic routing algorithm.
Zheng {\it et al.}~\cite{portal20mobihoc} proposed a transparent cross-technology opportunistic forwarding method
in low-power wireless networks to maximize the performance gain of using heterogeneous forwarders.
\cite{matrixcomp20cn} exploited the matrix decomposition to predict the probability of the node's contact
and optimized the delivery path from the source node to the destination node.
\cite{torp20trustcom} calculated the trust-based routing metric
according to the distributions of multiple attributes,
such as the distance, the hops, the degree and the residual energy.
\cite{ring20cn} achieved the balance between the energy consumption and the average delay
based on the virtual ring routing structure.
\cite{sporadic20wcnc} utilized the regular and sporadic pattern of the node's contact
to enhance the delivery rate in a cost-effective manner.

\subsection{Security \& Privacy in OppNets}
In order to preserve the location privacy of participants,
Du {\it et al.}~\cite{mixzone13tvt} generated virtual nodes
and conducted the pseudonym change in MixZones
with the incentives of the auction mechanism.
Zhang {\it et al.}~\cite{pp16tvt} designed a secure and privacy-preserving framework
to protect the routing metric based on the group signature, the homomorphic encryption
and Sakai�C-Ohgishi-�CKasahara (SOK) key agreement.
\cite{cntavd17infocom} proposed a contact avoidance routing protocol,
which quantified the probability that a message is delivered to its destination
without being compromised by adversaries.
Chen {\it et al.}~\cite{ppdtn17secon} protected the routing utilities and the forwarder information
via the commutative encryption technique and the order-preserving hashing scheme.
Through the real IDs of nodes are anonymized upon contacts,
the real ID-based encountering information is encrypted and collected
after the nodes disconnect in~\cite{face17ton}.
Based on $k$-anonymity, \cite{pploc18iotj} protected the vehicle's location privacy
by using virtual location and route confusion.
\cite{eprivo18tvt} proposed the similarity privacy mechanism,
which hid the neighbor list of nodes and obtained the common neighbors
based on the Paillier homomorphic encryption scheme.
\cite{ppoct18tvt} constructed a two-layer architecture,
which consisted of the Yao��s Millionaires-based routing decision
and the encryption-based message delivery process.
A trusted authority is exploited to calculate the routing utility
and protect the privacy information of nodes in~\cite{ppdtn19info}.
The routing algorithm in~\cite{ppsOppNet21cs}
provided the identity and location privacy for OppNets
with the public key sharing and the encryption.

\subsection{Differential Privacy}
Differential Privacy (DP)~\cite{Dwork14DP} has been a {\it de facto} standard for privacy protection.
The differential privacy methods in~\cite{purdue12asonam},
including node privacy, edge privacy and out-link privacy,
can preserve the privacy of the graph structure.
Bassily {\it et al.}~\cite{ldp15stoc} proposed the local differentially private (LDP) for succinct histograms.
Several LDP protocols were generalized and analyzed to choose the optimal parameter in~\cite{ldp17Usenix}.
RAPPOR in~\cite{rappor14ccs} estimated the client-side distributions of string-values
drawn from the discrete data dictionary via the bloom filter and the Permanent \& Instantaneous random responses.
Ren {\it et al.}~\cite{lopub18tifs} developed a local differentially private high-dimensional date publication algorithm
via the dimensionality reduction strategy and the EM\cite{EM77}-based distribution estimation method.
Since the pattern of time-series may be distorted by LDP,
Wang {\it et al.}~\cite{PatADP20info} sampled the remarkable points
and proposed the pattern-aware privacy preserving method based on Piecewise Linear Approximation (PLA).
Huang {\it et al.}~\cite{pbcn20tnsm} proposed a privacy preserving approach based on clustering and graph reconstruction
to disturb the released graph structure of the social network.
The personalized multiple optimized unary encoding is designed in \cite{pldp21scn} to
satisfy the personal privacy requirements.
\cite{dpweightg21adhnetw} protected the privacy of the edge weight and the frequent subgraph
in the weighted graph of the wireless network according to the differential privacy scheme.
In order to preserve the privacy of the time of the event occurrence,
the priority queue in wireless sensor network is applied the laplace noise in~\cite{tem20jnca}.
Considering that DP may change the optimality,
\cite{dpgame20ijcai} proposed the Privacy-Preserving Stackelberg Mechanism,
which obfuscates the gas demand profile exchanged by the gas market,
to also ensure that the outcomes of the resulting problem are close-to-optimality for each agent.
\cite{DPCon20ndss} introduced several post-processing steps
to achieve the consistency of the frequency estimation in LDP.
